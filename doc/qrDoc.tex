



%%
%% This is LaTeX2e input.
%%

%% The following tells LaTeX that we are using the 
%% style file amsart.cls (That is the AMS article style
%%
\documentclass{amsart}

%% This has a default type size 10pt.  Other options are 11pt and 12pt
%% This are set by replacing the command above by
%% \documentclass[11pt]{amsart}
%%
%% or
%%
%% \documentclass[12pt]{amsart}
%%

%%
%% Some mathematical symbols are not included in the basic LaTeX
%% package.  Uncommenting the following makes more commands
%% available. 
%%

\usepackage[utf8]{inputenc}
%\usepackage{amssymb}

%%
%% The following is commands are used for importing various types of
%% grapics.
%% 

%\usepackage{epsfig}  		% For postscript
%\usepackage{epic,eepic}       % For epic and eepic output from xfig

%%
%% The following is very useful in keeping track of labels while
%% writing.  The variant   \usepackage[notcite]{showkeys}
%% does not show the labels on the \cite commands.
%% 

%\usepackageshowkeys}


%%%%
%%%% The next few commands set up the theorem type environments.
%%%% Here they are set up to be numbered section.number, but this can
%%%% be changed.
%%%%

\newtheorem{thm}{Theorem}[section]
\newtheorem{prop}[thm]{Proposition}
\newtheorem{lem}[thm]{Lemma}
\newtheorem{cor}[thm]{Corollary}


%%
%% If some other type is need, say conjectures, then it is constructed
%% by editing and uncommenting the following.
%%

%\newtheorem{conj}[thm]{Conjecture} 


%%% 
%%% The following gives definition type environments (which only differ
%%% from theorem type invironmants in the choices of fonts).  The
%%% numbering is still tied to the theorem counter.
%%% 

\theoremstyle{definition}
\newtheorem{definition}[thm]{Definition}
\newtheorem{example}[thm]{Example}

%%
%% Again more of these can be added by uncommenting and editing the
%% following. 
%%

%\newtheorem{note}[thm]{Note}


%%% 
%%% The following gives remark type environments (which only differ
%%% from theorem type invironmants in the choices of fonts).  The
%%% numbering is still tied to the theorem counter.
%%% 


\theoremstyle{remark}

\newtheorem{remark}[thm]{Remark}


%%%
%%% The following, if uncommented, numbers equations within sections.
%%% 

\numberwithin{equation}{section}


%%%
%%% The following show how to make definition (also called macros or
%%% abbreviations).  For example to use get a bold face R for use to
%%% name the real numbers the command is \mathbf{R}.  To save typing we
%%% can abbreviate as

\newcommand{\R}{\mathbf{R}}  % The real numbers.

%%
%% The comment after the defintion is not required, but if you are
%% working with someone they will likely thank you for explaining your
%% definition.  
%%
%% Now add you own definitions:
%%

%%%
%%% Mathematical operators (things like sin and cos which are used as
%%% functions and have slightly different spacing when typeset than
%%% variables are defined as follows:
%%%

\DeclareMathOperator{\dist}{dist} % The distance.



%%
%% This is the end of the preamble.
%% 


\begin{document}

%%
%% The title of the paper goes here.  Edit to your title.
%%

\title{Ein numerisch stabiler Condensing Algorithmus für OCP}

%%
%% Now edit the following to give your name and address:
%% 

\author{Robert	Scholz}
\address{IWR Uni Heidelberg}
\email{robert.scholz@iwr.uni-heidelberg.de}
 % Delete if not wanted.

%%
%% If there is another author uncomment and edit the following.
%%

%\author{Second Author}
%\address{Department of Mathematics, University of South Carolina,
%Columbia, SC 29208}
%\email{second@math.sc.edu}
%\urladdr{www.math.sc.edu/$\sim$second}

%%
%% If there are three of more authors they are added in the obvious
%% way. 
%%

%%%
%%% The following is for the abstract.  The abstract is optional and
%%% if not used just delete, or comment out, the following.
%%%

\begin{abstract}
Bei der numerischen Lösung von OCPs treten dünnbesetzte strukturierte QPs auf. Es ist üblich diese Struktur mit der Hilfe eines Condensing Algorithmuses auszunutzen. Jedoch werden bei den bisher verwendeten Ansätzen viele Blockmatrizen miteinander multipliziert. Bei schlecht gestellten Problemen führt dies zu numerischen Problemen. Hier wird ein Ansatz vorgestellt, bei dem ausschließlich mit der Multiplikation von orthonormalen Matrizen gearbeitet wird.
\end{abstract}

%%
%%  LaTeX will not make the title for the paper unless told to do so.
%%  This is done by uncommenting the following.
%%

\maketitle

%%
%% LaTeX can automatically make a table of contents.  This is done by
%% uncommenting the following:
%%

%\tableofcontents

%%
%%  To enter text is easy.  Just type it.  A blank line starts a new
%%  paragraph. 
%%
\section{Einleitung}
Bei der Lösung von Optimale Steurungs Problemen (OCP) mit SQP-Verfahren tauchen quadratische Programme (QP)  mit einer typischen Struktur auf. Durch die Ausnutzung dieser Struktur lässt sich die Geschwindigkeit der Verfahren stark erhöhen. Dabei wird das ursprüngliche dünnbesetzte QP in ein dichtbesetztes QP mit kleinerer Dimension transformiert. In diesem Bericht stellen wir einen Ansatz vor, der auf QR-Faktorisierung beruht. Da nur mit orthogonalen Matrizen gearbeitet wird zeichnet dieser Ansatz sich durch numerische Stabiltät aus.

Wir behandeln quadratische Programme mit der folgenden Strukur. Dabei gibt $n_u$ die Anzahl der Steuerung, $n_s$ die Anzahl der Zustande und $n_{dis}$ die Anzahl der Shootingintervalle an.

\begin{align*}
\underset{u,s,p}{max}\ \	&p^\top B_{pp} p+f_p^\top p  \\ 
						&+\frac{1}{2}\sum_{i=1}^{n_{dis}} s_i^\top B_{ss}^i s_i+u_i^\top B_{uu}^i u_i \\
						&+\sum_{i=1}^{n_{dis}} s_i^\top B^i_{su} u_i+p^\top B_{ps}^i s_i +p^\top B_{pu}^i u_i  \\
						&+\sum_{i=1}^{n_{dis}} (f_u^i)^\top u_i+(f_s^i)^\top s_i \\
s.t. 	\ \				&s_{i+1}=X_p^i p +X_s^i s_i + X_u^i u_i -X_c^i  		&& i=1,\hdots,n_{dis}-1\\
						& R_u^i u_i + R_s^i s_i+R_p^i p =R_c   		&& i=1,\hdots,n_{dis} \\
						& \sum_{i=1}^{n_{dis}} C_u^i u_i + C_s^i s_i + R_p^i p = C_c \\
						& p_l \leq p \leq p_u \\
						& s_l^i \leq s_i \leq s_u^i && i=1,\hdots,n_{dis} \\ 
						& u_l^i \leq u_i  \leq u_u^i  && i=1,\hdots,n_{dis}
\end{align*}

Die Beschreibung des Algorithmus in diesem Bericht orientiert sich an der dazu gehörigen Matlab Implementierung. Dabei wurde der Programmcode nicht auf Geschwindigkeit optimiert. Stattdessen ist der Fokus auf der Übersichtlich und der Nachvollziehbarkeit.
\section{Der Algorithmus}

Analog zu der Implementierung werden wir den Algorithmus in vier Schritten darstellen.

\subsection*{Schritt 1: Anordnung der einzelnen Matrizen}

Zunächst definieren wir $x=(p,s_0,u_0,s_1,u_1,\hdots,s_{n_{dis}},u_{n_{dis}})$. Somit können wir das QP auch schreiben als:

\begin{align*}
\underset{x}{max} \ \ 	& \frac{1}{2}	x^\top H x+f^\top x \\
s.t. \ \ 				& Sx=s \\
						& Qx=q \\
						& b_l \leq x \leq b_u \\ 
\end{align*}

Dabei ist $$H= \left[ \begin{array}{cccccc}
B_{pp} & B_{ps}^1 & B_{pu}^1 & \hdots & B_{ps}^{n_{dis}} & B_{pu}^{n_{dis}} \\ 
(B_{ps}^1)^\top & B_{ss}^1 & B_{su}^1 &  &  & \\ 
(B_{pu}^1)^\top & (B_{su}^1)^\top & B_{uu}^1 &  &  &  \\ 
\vdots &  &  & \ddots &  &  \\ 
(B_{ps}^{n_{dis}})^\top &  &  &  & B_{ss}^{n_{dis}} & B_{su}^{n_{dis}} \\ 
(B_{pu}^{n_{dis}})^\top &  &  &  & (B_{su}^{n_{dis}})^\top & B_{uu}^{n_{dis}}
\end{array} \right]  
,f=\left[ \begin{array}{c}
f_p \\ 
f_s^1 \\ 
f_u^1 \\ 
\vdots \\ 
f_s^{n_{dis}} \\ 
f_u^{n_{dis}}
\end{array}  \right]$$

$$S=\left[\begin{array}{cccccccc}
X_p^0 & X_s^0 & X_u^0 & - I &  &  &  &  \\ 
X_p^1 &  &  & X_s^1 & X_u ^1  & -I &  &  \\ 
\vdots &  &  &  & \ddots & \ddots & \ddots &  \\ 
X_p^{n_{dis}-1} &  &  &  &  & X_s^{n_{dis}-1} & X_u^{n_{dis}-1} & -I
\end{array}  \right],
s=\left[ \begin{array}{c}
X_c^ 0 \\ 
X_c^1 \\ 
\hdots \\ 
X_c^{n_{dis}-1}
\end{array} \right] $$

$$Q=\left[ \begin{array}{cccccccc}
R_p^0 & R_s^0 & R_u^0 &  &  &  &  &  \\ 
R_p^1 &  &  & R_s^1 & R_u^1 &  &  &  \\ 
\vdots &  &  &  &  & \ddots &  &  \\ 
R_p^{n_{dis}} &  &  &  &  &  & R_s^{n_{dis}} & R_u^{n_{dis}} \\ 
C_p & C_s^0 & C_u^0 & C_s^1 & C_u^1 & \hdots & C_s^{n_{dis}} & C_u^{n_{dis}}
\end{array}  \right],
q= \left[ \begin{array}{c}
R_c^0 \\ 
R_c^1 \\ 
\vdots \\ 
R_c^{n_{dis}} \\ 
C_c
\end{array}  \right]$$

$$u= \left[ \begin{array}{c}
p_l \\ 
s^0_l \\ 
u^0_l \\ 
\vdots \\ 
s^{n_{dis}}_l \\ 
u^{n_{dis}}_l
\end{array} \right],
l= \left[ \begin{array}{c}
p_u \\ 
s^0_u \\ 
u^0_u \\ 
\vdots \\ 
s^{n_{dis}}_u \\ 
u^{n_{dis}}_u
\end{array} \right]$$

\subsection*{Schritt 2: Iterative QR-Faktorrisierung und Variabelentransformation}

Wir betrachten nun die einzelnen Matchingblöcke und berechnen eine entsprechende QR-Zerlegung. 

Sei $Q_0$ und $R_0$ die QR-Zerlegung des Blockes $[X_s^0 X_u^0 -I]$, wobei $Q_0$ eine orthonormale Matrix der Dimension $n_u+2n_s$ ist und $R_0$ eine invertierbare untere Dreiecksmatrix mit der Dimension $n_s$ ist.

Es gilt also :

$$\left[X_s^0 \  X_u^0 \ -I \right]= \left[R_0 \ 0 \right]Q_0$$

und somit ist :

$$S=\left[\begin{array}{cccccccc}
X_p^0 & R_0 & 0 & 0 &  &  &  &  \\ 
X_p^1 & \tilde{X_s^0} & \tilde{X_u^0} & \tilde{X_s^1} & X_u ^1  & -I &  &  \\ 
\vdots &  &  &  & \ddots & \ddots & \ddots &  \\ 
X_p^{n_{dis}-1} &  &  &  &  & X_s^{n_{dis}-1} & X_u^{n_{dis}-1} & -I
\end{array}  \right] \left[\begin{array}{cccc}
Q_0 &  &  &  \\ 
 & I &  &  \\ 
 &  & \ddots &  \\ 
 &  &  & I
\end{array}  \right]$$



\end{document}







